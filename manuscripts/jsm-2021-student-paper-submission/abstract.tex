  Log scales are often used to display data over several orders of magnitude within one graph. During the COVID-19 pandemic, we have seen both the benefits and the pitfalls of using log scales to display case counts, transmission rates, and outbreak regions. In this paper, we explore the use of linear and log scales to determine whether our ability to notice differences in the data is different when different scales are used. Our goal is to provide basic research to support the principles used to guide design decisions in scientific visualizations of exponential data. We found that displaying increasing exponential data on a log scale improved the accuracy of differentiating between models with slight curvature differences, particularly when identifying an exponential curve with a lower growth rate than others. When there was a larger curvature difference, participants accurately differentiated between the two curves on both the linear and log scale. An exception occurs when identifying a plot with more curvature than the surrounding plots, supporting \cite{best_perception_2007} whose results found that accuracy was higher when nonlinear trends were presented indicating that it is hard to say something is linear (i.e. something has less curvature), but easy to say that it isn't (i.e. something has more curvature).
