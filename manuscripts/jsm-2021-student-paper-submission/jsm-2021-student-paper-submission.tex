% interactcadsample.tex
% v1.03 - April 2017

\documentclass[]{interact}

\usepackage{epstopdf}% To incorporate .eps illustrations using PDFLaTeX, etc.
\usepackage{subfigure}% Support for small, `sub' figures and tables
%\usepackage[nolists,tablesfirst]{endfloat}% To `separate' figures and tables from text if required

\usepackage{natbib}% Citation support using natbib.sty
\bibpunct[, ]{(}{)}{;}{a}{}{,}% Citation support using natbib.sty
\renewcommand\bibfont{\fontsize{10}{12}\selectfont}% Bibliography support using natbib.sty

\theoremstyle{plain}% Theorem-like structures provided by amsthm.sty
\newtheorem{theorem}{Theorem}[section]
\newtheorem{lemma}[theorem]{Lemma}
\newtheorem{corollary}[theorem]{Corollary}
\newtheorem{proposition}[theorem]{Proposition}

\theoremstyle{definition}
\newtheorem{definition}[theorem]{Definition}
\newtheorem{example}[theorem]{Example}

\theoremstyle{remark}
\newtheorem{remark}{Remark}
\newtheorem{notation}{Notation}

% see https://stackoverflow.com/a/47122900


\usepackage{hyperref}
\usepackage[utf8]{inputenc}
\def\tightlist{}
\usepackage[usenames,dvipsnames]{color}
\newcommand{\er}[1]{\textcolor{Orange}{#1}}
\newcommand{\svp}[1]{\textcolor{Green}{#1}}
\newcommand{\rh}[1]{\textcolor{Plum}{#1}}


\begin{document}

\articletype{JSM 2021 Student Paper Award (ASA sections on Statistical
Computing and Statistical Graphics)}

\title{Perception of exponentially increasing data displayed on a log
scale}


\author{\name{Emily A. Robinson$^{a}$, Reka Howard$^{a}$, Susan
VanderPlas$^{a}$}
\affil{$^{a}$Department of Statistics, University of Nebraska -
Lincoln,}
}

\thanks{CONTACT Emily A.
Robinson. Email: \href{mailto:emily.robinson@huskers.unl.edu}{\nolinkurl{emily.robinson@huskers.unl.edu}}, Reka
Howard. Email: \href{mailto:rekahoward@unl.edu}{\nolinkurl{rekahoward@unl.edu}}, Susan
VanderPlas. Email: \href{mailto:susan.vanderplas@unl.edu}{\nolinkurl{susan.vanderplas@unl.edu}}}

\maketitle

\begin{abstract}
Log scales are often used to display data over several orders of
magnitude within one graph. During the COVID pandemic, we've seen both
the benefits and the pitfalls of using log scales to display data. This
paper aims to\ldots{}
\end{abstract}

\begin{keywords}
Exponential; Log; Visual Inference; Perception
\end{keywords}

\textcolor{Green}{Emily and Reka, when we get into the editing stage, I've found this strategy to be useful: basically, when you add new text, use your color (feel free to change the command, for now yours are set to \textcolor{Orange}{$\backslash$er} and \textcolor{Plum}{$\backslash$rh}.}
\textcolor{Green}{The way this usually works is that when e.g. I read over a document that Emily has recently edited, I will remove her flagged text to indicate that I've seen/accepted the changes (and vice versa - I'll edit text and highlight it with my color, and you can accept/modify and flag yours too) -- sometimes modifications happen first and then all of the color in a paragraph gets taken out once we've moved on.}
\textcolor{Green}{This not only leads to a nice rainbow effect, but you can quickly spot changes, too. If you're changing some slight phrasing/wording that doesn't change meaning, it's not necessary to highlight those changes - highlight content changes, not e.g. verb tenses.}
\textcolor{Green}{If something is a comment and has been addressed, comment it out initially and then delete the line after a couple of weeks.}

\hypertarget{introduction}{%
\section{Introduction}\label{introduction}}

\citep{buja_statistical_2009, vanderplas_clusters_2017}

\hypertarget{data-generation}{%
\section{Data Generation}\label{data-generation}}

\hypertarget{model-generation-and-simulation}{%
\subsection{Model Generation and
Simulation}\label{model-generation-and-simulation}}

\hypertarget{parameter-selection}{%
\subsection{Parameter Selection}\label{parameter-selection}}

\hypertarget{study-design}{%
\section{Study Design}\label{study-design}}

\hypertarget{lineup-setup}{%
\subsection{Lineup Setup}\label{lineup-setup}}

\hypertarget{participant-recruitment}{%
\subsection{Participant Recruitment}\label{participant-recruitment}}

\hypertarget{task-description}{%
\subsection{Task Description}\label{task-description}}

\hypertarget{results}{%
\section{Results}\label{results}}

\hypertarget{effect-of-curvature}{%
\subsection{Effect of Curvature}\label{effect-of-curvature}}

\hypertarget{effect-of-variability}{%
\subsection{Effect of Variability}\label{effect-of-variability}}

\hypertarget{linear-vs-log}{%
\subsection{Linear vs Log}\label{linear-vs-log}}

\hypertarget{participant-reasoning}{%
\subsection{Participant Reasoning}\label{participant-reasoning}}

\hypertarget{discussion}{%
\section{Discussion}\label{discussion}}

\hypertarget{conclusion}{%
\subsection{Conclusion}\label{conclusion}}

\hypertarget{future-research}{%
\subsection{Future Research}\label{future-research}}

\hypertarget{supplementary-materials}{%
\section*{Supplementary Materials}\label{supplementary-materials}}
\addcontentsline{toc}{section}{Supplementary Materials}

\hypertarget{acknowledgements}{%
\section*{Acknowledgement(s)}\label{acknowledgements}}
\addcontentsline{toc}{section}{Acknowledgement(s)}

\bibliographystyle{tfcad}
\bibliography{references.bib}




\end{document}
